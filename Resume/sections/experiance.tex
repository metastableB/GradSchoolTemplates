\section{Research Experience}
\begin{project}
  \title{Faster Recurrent Networks: Feed-forward Approximations and Rolling Predictions}
  \supervisor{Advisors: Dr. Prateek Jain \& Prof. Venkatesh Saligrama}
  \duration{Ongoing}
  \location{\color{verydarkgray}{Microsoft Research}}
  \description{Exploring feed-forward approximations and rolling predictions for efficient RNN inference. Inspired by recent results that show that RNNs are well approximated by feed-forward networks in training and inference.} 
\end{project}
\begin{project}
  \title{Object Detection for Resource Constrained Devices}
  \supervisor{Advisors: Dr. Prateek Jain \& Prof. Venkatesh Saligrama}
  \duration{Ongoing}
  \location{Microsoft Research}
  \description{Devising new computer vision techniques that can enable object detection on resource constrained devices. Current state of the art techniques have large working memory and compute requirements making them unsuitable for resource constrained devices.} 
\end{project}
\begin{project}
  \title{Multiple Instance Learning For Fast and Accurate Sequential Data Classification}
  \supervisor{Advisors: Dr. Prateek Jain \& Dr. Harsha Simhadri}
  \duration{Jan - May '18}
  \link{\href{https://dkdennis.xyz/publications/}{[Preprint]}}
  \location{Microsoft Research}
  \description{Developed a multiple-instance-learning based algorithm, EMI-RNN, that recovers the distinguishing signature of minimum length for each class in time series classification. Smaller signatures result in lower computational costs and effective use of classification model's capacity thereby improving performance while reducing compute by up to 72x. For nice data, showed linear convergence to global optimum in the number of non-noise samples in a non-homogeneous setting. \textit{\cusemph{(Accepted at NIPS '18)}}}
\end{project}
\clearpage
\begin{project}
  \title{Machine Learning Based Gesture Recognition on Resource Constrained Devices}
  \supervisor{Advisors: Dr. Prateek Jain, Dr. Harsha Simhadri \& Dr. Manik Varma}
  \duration{July - Dec, '17}
  \location{Microsoft Research}
  \link{\href{https://dkdennis.xyz/publications/}{[Preprint]}}
  \description{Developed an efficient machine learning pipeline to enable \textit{GesturePod}, a low resource microcontroller based device, to perform robust, low-latency gesture recognition. The ProtoNN algorithm powered prediction pipeline along with communication and storage stack works under 32kB RAM on a 48MHz processor.\\\textit{\cusemph{(In submission, CHI '19 \& Microsoft's demonstration at NIPS '18)}}.}
\end{project}
\begin{project}
  \title{Keyword Spotting in Low Resource Settings}
  \supervisor{Advisors: Dr. Prateek Jain \& Dr. Harsha Simhadri}
  \duration{Nov '17 - Sep '18}
  \location{Microsoft Research}
  \description{Developed a small, fast and accurate classifier based on LSTM and ProtoNN to enable real-time keyword spotting on Raspberry Pi3. Developed EMI-RNN to make it possible on even smaller devices (Raspberry Pi0, MXChip). \textit{\cusemph{(Demonstration part of NIPS '18)}}.
  }
\end{project}
\begin{project}
  \title{Talk-Bot: Federated Human Detection for Collaborative Multi-angle Videography}
  \supervisor{Advisors: Dr. Arijit Mondal \& Dr. Jimson Mathew}
  \duration{Oct - Dec '17}
  \location{IIT Patna}
  \link{\href{https://dkdennis.xyz/projects/\#Talk-Bot}{[Prototype]}}
  \description{Developed a cluster of Raspberry Pi3s with a computer vision stack that collaborate with each other in real time to track a presenter so as to provide a multi-angle video stream to be used for cost efficient live streaming of talks. \textit{\cusemph{(Runner up at Grand Challenge, ISED '16)}}
  }
\end{project}
\begin{project}
  \title{Nagging Naagin: The Q-Learning Snake}
  \supervisor{Advisor: Dr. Arijit Mondal}
  \duration{Feb - April '17}
  \location{IIT Patna}
  \link{\href{https://dkdennis.xyz/projects/\#NaggingNaagin}{[Demo, GitHub \& Report]}}
  \description{Taught an agent to play the classic game \textit{Snake} through reinforcement learning. Created a custom version of the game to allow for a multi-bandit formulation (snake, adversarial food placement). Implemented and analyzed various search and RL algorithms --- reflex agents, min-max and expectimax trees, Q-learning and approximate Q-learning with DQN.}
\end{project}
\begin{project}
  \title{RISC-V Micro-architecture Processor for Embedded Devices}
  \supervisor{Advisor: Dr. Arijit Mondal}
  \duration{Thesis project}
  \location{IIT Patna}
    \link{\href{https://dkdennis.xyz/projects/\#RISCV}{[GitHub \& Publication]}}
  \description{Developed a RISC-V based single cycle micro architecture processor optimized for low-cost embedded devices, a bare bones simulator and an FPGA prototype. Additionally wrote a custom assembler-linker-loader tool chain to run native programs on the prototype.\textit{\cusemph{(Published at ISED '17)}}}
\end{project}